\chapter{Applications of Mathematical Methods}

\section{Fundamentals of Translational and Rotational Systems}

Damper systems do not have equilibrium but spring systems do.
The number odes needed to be solved is based on the number of degrees of 
freedom in the system.

%\begin{tikzpicture}[every node/.style={outer sep=0pt},thick,
% mass/.style={draw,thick},
% spring/.style={thick,decorate,decoration={zigzag,pre length=0.3cm,post
% length=0.3cm,segment length=6}},
% ground/.style={fill,pattern=north east lines,draw=none,minimum
% width=0.75cm,minimum height=0.3cm},
% dampic/.pic={\fill[white] (-0.1,-0.3) rectangle (0.3,0.3);
% \draw (-0.3,0.3) -| (0.3,-0.3) -- (-0.3,-0.3);
% \draw[line width=1mm] (-0.1,-0.3) -- (-0.1,0.3);}]
%
%  \node[mass,minimum width=3.5cm,minimum height=2cm,fill=green!50!black] (m1) {$m_1$};
%  \node[mass,minimum width=3.5cm,minimum height=2cm,fill=green!50!black,right=1.5cm of
%  m1] (m2) {$m_2$};
%  \node[mass,minimum width=2.8cm,minimum height=1.5cm,fill=red!70,right=1.5cm of
%  m2] (ma) {$m_a$};
%  \node[left=2cm of m1,ground,minimum width=3mm,minimum height=2.5cm] (g1){};
%  \draw (g1.north east) -- (g1.south east);
%
%  \draw[spring] ([yshift=3mm]g1.east) coordinate(aux)
%   -- (m1.west|-aux) node[midway,above=1mm]{$k_1$};
%  \draw[spring]  (m1.east|-aux) -- (m2.west|-aux) node[midway,above=1mm]{$k_2$};
%  \draw[spring]  (m2.east|-aux) -- (ma.west|-aux) node[midway,above=1mm]{$k_a$};
%
%  \draw ([yshift=-3mm]g1.east) coordinate(aux')
%   -- (m1.west|-aux') pic[midway]{dampic} node[midway,below=3mm]{$c_1$}
%     (m1.east|-aux') -- (m2.west|-aux') pic[midway]{dampic} node[midway,below=3mm]{$c_2$}
%     (m2.east|-aux') -- (ma.west|-aux') pic[midway]{dampic} node[midway,below=3mm]{$c_a$};
%
%  \foreach \X in {1,2}  
% {\draw[thin] (m\X.north) -- ++ (0,1) coordinate[midway](aux\X);
% \draw[latex-] (aux\X) -- ++ (-0.5,0) node[above]{$F_\X$}; 
%   \draw[thin,dashed] (m\X.south) -- ++ (0,-1) coordinate[pos=0.85](aux'\X);
%   \draw[latex-] (aux'\X) -- ++ (-1,0) node[midway,above]{$x_\X$}
%    node[left,ground,minimum height=7mm,minimum width=1mm] (g'\X){};
%   \draw[thick] (g'\X.north east) -- (g'\X.south east);
%  }
%
%  \draw[thin,dashed] (ma.south) -- ++(0,-1.2);
%  \draw[latex-] (ma.south |- aux'1) -- ++ (-1.5,0) coordinate(aux3)
%  node[midway,above]{$x_d$};
%  \draw[thin] ([yshift=-2mm]aux3) |- ++ (-1,0.5) -| (m2.-40);
%\end{tikzpicture}
\begin{center}
\begin{circuitikz}
    \pattern[pattern=north east lines] (0,0) rectangle (7,.25);
    \draw[thick] (0,.25) -- (7,.25);

    \draw (3, .25) to[spring, l=$k$] (3,2);
    \draw (4, .25) to[damper, l_=$b$] (4,2);
    \draw[fill=gray!40] (2.5,2) rectangle (4.5, 3);
    \node at (3.5, 2.5) {$m$};
    \draw[thick, ->] (3.5, 4) -- (3.5, 3);
    \node at (3.75, 3.75) {$F$};
\end{circuitikz}
\end{center}

A \alert{stiffness element} is a spring as shown below. A spring with an 
unstretched length of $l_0$ has a length of $$l(t) = l_0 + x_2(t)-x_1(t)$$ with
the application of the shown forces in the figure below. For a linear spring
$F = k(x_1-x_1)$, where $k$ is the \alert{stiffness coefficient}. Simplifying 
this gives $F = kx(t)$ or just $F=kx$.

\begin{center}
\begin{circuitikz}
    \draw (3,0) to[spring, l=$k$] (5,0);
    \draw[thick, ->] (5,0) -- (5.5,0);
    \node at (5.25,0.5) {$F$};
    \draw[thick, ->] (3,0) -- (2.5,0);
    \node at (2.75,0.5) {$F$};
\end{circuitikz}
\end{center}
