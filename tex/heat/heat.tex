\chapter{Heat Transfer and Transport}

We begin by introducting some fundamental equations required.
\begin{theorem}[Fundamental Equation of accumulation]
    $A = I - O + G - C$
    \begin{align*}
        \text{Accumulation}~&=~\text{In}~-~\text{Out}~+~\text{Generative}~-~\text{Consumption}
    \end{align*}
\end{theorem}
\begin{theorem}[Fourier's Law]
    $\frac{Q}{A} = q = -\frac{k \dif T}{\dif x} ~\text{or}~-\frac{k \dif T}{\dif r}$,
    where $Q$ is the total heat, $A$ is the area, and $q$ is the heat flux.
    $q=h(T_1-T_2)$, where $h$ is the heat transfer coefficient and $T_1-T_2$ is
    the change in the temperature.
\end{theorem}

\begin{example}
    Consider a slab going into the paper with a width of $w$, as shown below.

    Heat moves away from $x=0$, and the interior has $.$
\end{example}
Since we assume the system to be stead state, the accumulation is equal to $0$.
Thus, $0 = I-O+G-C$. $I=q(x)wL$, $O=q_i(x+\Delta x)wL$, and $G=+S_0wL\Delta x$.
Plugging in and simplifying gives $$0=q_i(x)-q_i(x+\Delta x)+S_0 \Delta x.$$
Dividing by $\Delta X$ and taking a limit as $\Delta x$ approaches $0$, we can obtain
$$\lim_{\Delta x \rightarrow 0} \frac{q_i(x+\Delta x) - q_i(x)}{\Delta x} = S_0.$$
This is the limit definition of a derivative. Therefore we have
\begin{align*}
    \frac{\dif q_i}{\dif x} = S_0 &\implies \int \dif q_i = \int S_0 \dif x \\
                                  &\implies q_i(x) = S_0x+c_1
\end{align*}

\begin{example}
    Consider a cylindrical slab going into the paper with a length $L$, an
    inner radius of $R_1$, an outer radius of $R_2$. Same initial conditions
    as prior example.
\end{example}
We notice first that $q = h(T_s-T_f)$, where $T_s$ is the temperature on the
surface and $T_f$ is the temperature on the outside. Additionally, since
we assume the system to be steady state, the accumulation is equal to $0$. Firstly,
we proceed to solve for the heat flux of the inner material. We have
\begin{align*}
    I &= q_i(x)2\pi r L \\
    O &= q_i(x)(r+\Delta r)2 \pi (r+\Delta r)L \\
    G &= S_0\pi L ((r+ \Delta r)^2-r^2) \\
    C &= 0.
\end{align*}
Thus, since the accumulation is equal to $0$,
$$q_i(x)2\pi r L  -  q_i(x)(r+\Delta r)2 \pi (r+\Delta r)L + S_0\pi L ((r+ \Delta r)^2-r^2) = 0.$$
After simplification, rearrangment  and taking a limit, we attain the following equation
$$\lim_{\Delta r \rightarrow 0} \frac{q_i(r+\Delta r)(r+\Delta r)-q_i(r)r}{\Delta r}
= \lim_{\Delta r \rightarrow 0} S_0 \left(r+\frac{\Delta r}{2}\right).$$
Using the limit definition of the derivative, this simplifies to
$$\frac{\dif}{\dif r} (q_i(r)r) = S_0 r.$$ Therefore,
$$q_i(r)r = \frac{1}{2}S_0 r^2 + c_1.$$ Since $q_i(0) = 0$, $c_1 = 0$, giving
$$q_i(r) = \frac{1}{2}S_0 r.$$

Now, we proceed to solve for the heat flux of the cladding. We have
\begin{align*}
    I &= q_c(x)2 \pi r L \\
    O &= q_c(x)(r+\Delta r)2 \pi (r+\Delta r) L.
\end{align*}
So, since the accumulation is equal to $0$, $$q_c(x)2 \pi r L - q_c(x)(r+\Delta r)2 \pi (r+\Delta r) L = 0.$$
Simplifying, rearranging, and taking a limit we obtain
\begin{align*}
    \lim_{\Delta r \rightarrow 0} \frac{q_c(r+\Delta r)(r+\Delta r)-q_c(r)r}{\Delta r} = 0
    \implies \frac{\dif}{\dif r} q_c(r)r = 0.
\end{align*}
Solving this differential equation, $q_c(r)r = c_2$. Since $q_i(R_1) = q_c(R_2)$
and $q_i(R_1) = \frac{1}{2}S_0 R_1$, $c_2 = \frac{1}{2}S_0 R_1^2$. Thus,
$$q_c(r) = \frac{S_0R_1^2}{2r}.$$

By Fourier's law, 
\begin{align*}
    q_c = \frac{-K_I \dif T_c}{\dif r} &\implies \frac{S_0R_1^2}{2r} = \frac{-K_I \dif T_c}{\dif r} \\
                                       &\implies \int \frac{S_0R_1^2}{2}\frac{1}{r} \dif r = \int -K_I \dif T_c \\
                                       &\implies c_3 + \frac{S_0R_1^2}{2}\ln r = -K_I T_c(r) \\
                                       &\implies T_c(r) = -\frac{S_0R_1^2}{2K_I}\ln r + c_3.
\end{align*}

We then use the fact that $q_c = h(T_s-T_f) \implies q_c(R_2) = h(T_c(R_2)-T_f)$.
This is equivalent to
$$\frac{S_0R_1^2}{2R_2} = h\left( -\frac{S_0R_1^2}{2K_I} \ln R_2 +c_3 - T_f\right)
\implies c_3 = \frac{S_0R_1^2}{2R_2h} +\frac{S_0R_1^2}{2K_I} \ln R_2 +T_f$$



