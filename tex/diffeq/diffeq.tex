\chapter{Ordinary Differential Equations}
\section{Separable Differential Equations}

A separable differential equation can be simplified to the form:
$$g(y)y'=f(x) \implies g(y)\dif y = f(x) \dif x,$$
where $f$ and $g$ are functions of $x$ and $y$ respectively. Letting
$G'(y)=g(y)$ and $F'(x)=f(x)$ and taking the integral of both sides of the
differential equation, we get $$G(y) = F(x)+C$$ for some constant $C$. Using
a separation of variables is most helpful as a first try when attempting
first-order linear ordinary differential equations. We go more in detail later
in the chapter.

%\begin{example}
%    Find $y$ for the following differential equation:
%    $$\frac{dy}{dx} = x.$$
%\end{example}
%\begin{soln}
%    We see that $dy = x ~ dx$. Taking the integral of both sides, we obtain
%    $$y = \frac{1}{2} x^2 + C.$$
%\end{soln}
%
%\begin{example}
%    Find $y$ for the following differential equation:
%    $$\frac{dy}{dx} = \frac{1}{x}.$$
%\end{example}
%\begin{soln}
%    We see that $dy = \frac{1}{x} ~ dx$. Taking the integral of both sides,
%    we obtain $$y = \ln|x| + C.$$
%\end{soln}
%\begin{example}
%    Find $y$ for the following differential equation:
%    $$\frac{dy}{dx} = \cos x.$$
%\end{example}
%\begin{soln}
%    We see that $dy = \cos x ~ dx$. Taking the integral of both sides, we obtain
%    $$\boxed{y = \sin x + C}.$$
%\end{soln}
%\begin{example}
%    Find $y$ for the following differential equation:
%    $$\frac{dy}{dx} = e^x.$$
%\end{example}
%\begin{soln}
%    We see that $dy = e^x ~ dx$. Taking the integral of borth sides, we obtain
%    $$\boxed{y =  e^x + C}.$$
%\end{soln}
%\begin{example}
%    Find $y$ for the following differential equation: 
%    $$\frac{dy}{dx} = xy.$$
%\end{example}
%\begin{soln}
%    We see that $\frac{1}{y} ~ dy  = x ~ dx$. Taking the integral of
%    both sides, we get
%    $$\ln |y| = \frac{1}{2}x^2 + C \implies y = e^{\frac{1}{2}x^2+C}.$$
%\end{soln}
\begin{example}
    Find $y$ for the following differential equation:
    $$\frac{dy}{dx} = x^2y-2xy.$$
\end{example}
\begin{soln}
    Separating the variables, we see that $\frac{1}{y} \dif y = (x^2-2x) \dif x$.
    Taking the integral of both sides we get that
    $$\ln |y| = \frac{1}{3}x^3-x^2 + C \implies y = Ce^{\frac{1}{3}x^3-x^2}.$$
\end{soln}
%\begin{example}
%    Find $y$ for the following differential equation:
%    $$(x^2+x)~\frac{dy}{dx} = y-1.$$
%\end{example}
%\begin{soln}
%    We see that $\frac{1}{y-1} ~ dy = \frac{1}{x^2+x} ~ dx$. Taking the
%    integral of both sides, we find that
%    $$\ln|y-1| = \ln|x|-\ln|x+1| +C\implies y = C\left|\frac{x}{x+1}\right|+1.$$
%\end{soln}
\begin{question}
    What is $y$ when given the initial condition $y(0)=e$?
\end{question}
%\begin{soln}
%    By a previous example we have $$y = C\left(e^{\frac{1}{3}x^3-x^2}\right).$$
%    By the initial value, we solve for $C$ to get $C = e$. Therefore,
%    $$\boxed{y = e^{\frac{1}{3}x^3-x^2+1}}.$$
%\end{soln}

\section{Homogeneous Ordinary Differential Equations}
The general from of a linear ordinary differential equation(ODE) is
$$a_ny^{(n)} +a_{n-1}y^{(n-1)}+\cdots+a_1y'+a_0y = f(x),$$
where $a_i \in \RR[x]$ for $i \in \{0, 1, \dots, n\}$ and $y^{(j)}$ denotes
the $j$th derivative of $y$. It is then said that this ODE
has \alert{order} $n$ since it is the highest derivative in the equation.
\begin{example}[ODE]
    $$\frac{d^2f}{dx^2} + 3 \frac{df}{dx} + 5 = 0  \implies y''+3y'+5=0.$$
\end{example}
\begin{remark}
    Partial differential equations also exist.
\end{remark}
%\begin{example}[PDE]
%    $$\frac{\partial^2f}{\partial x^2} + \frac{\partial^2 f}{\partial y^2} = 0.$$
%\end{example}

\begin{definition}
    A \alert{homogeneous} ordinary differential equation is one with $f(x) = 0$ 
    and has constant coefficients.
\end{definition}
\begin{example}[Homogeneous ODE]
    Consider the differential equation $$y''+7y'+12y=0.$$ Functions of the
    form $y=e^{mx}$ are an educated guess for solving homoegeneous ODEs. So,
    letting $y=e^{mx}$ means $y' = me^{mx}$ and $y''= m^2e^{mx}$. Plugging
    these in our differential equation,
    \begin{align*}
        m^2e^{mx}+7me^{mx}+12e^{mx}&=0 \\
        e^{mx}\left(m^2+7m+12\right)&=0 \\
        e^{mx}(m+4)(m+3) &= 0.
    \end{align*}
    Therefore, $m=-4,-3$, implying that $y_1=C_1e^{-4x}$ and $y_2=C_2e^{-3x}$. 
    This means that our solution of the differential
    equation is $$y_h = y_1+y_2 =  C_1e^{-4x} + C_2e^{-3x}.$$
\end{example}
At first, it might be puzzling why we consider $y=e^{mx}$ to be an educated
guess for solving homogeneous ODEs with constant coeficients. However, let us
consider the following first-order differential equation 
$$y'+ky=0 \implies y' = -ky.$$ Here, it is natural to guess that $y=e^{mx}$
because $e^{mx}$ is the only function where its derivative is equal to some
scalar multiple of itsself. So, for other homogeneous ODEs with constant
coefficient and of higher degree, we also use the guess as a means
of finding the general solution.
\begin{exercise}[Homogeneous ODE with initial values]
    $$y''+y'-2y=0,~y(0) = 4,~y'(0)=-5.$$
\end{exercise}
%\begin{soln}
%    Let $y=e^{mx}$. This means $y'=me^{mx}$ and $y''=m^2e^{mx}$. Plugging in,
%    we get
%    \begin{align*}
%        m^2e^{mx}+me^{mx}-2e^{mx} &= 0\\
%        e^{mx} \left(m^2+m-2\right) &= 0 \\
%        e^{mx}(m-1)(m+2) &=0.
%    \end{align*}
%    So, $m=-2,1$. Therefore $y_h = C_1e^{-2x} + C_2e^{x}$. Then, we get the 
%    following system from the initial values:
%    \begin{align*}
%         C_1+C_2 &= 4 \\
%         -2C_1+C_2&=-5.
%    \end{align*}
%    Solving, we get $C_1 = 3$ and $C_2 = 1$. So, $\boxed{y_h = 3e^{-2x}+e^{x}}$.
%\end{soln}
Now, we consider what happens if we get a polynomial with repeated roots or
if we are already given a solution to the differential equation.
\newpage
\begin{example}[Homogeneous ODE with repeated roots]
    $$y''+4y'+4y=0.$$
    Let $y=e^{mx}$. This means $y'=me^{mx}$ and $y''=m^2e^{mx}$. PLugging in, 
    we get
    \begin{align*}
        m^2e^{mx}+4me^{mx}+4e^{mx} &= 0 \\
        e^{mx}(m+2)^2 &= 0.
    \end{align*}
    Thus, $\boxed{y_1 = C_1e^{-2x}}$. We now proceed to use the  \alert{reduction of order} 
    method.
    So, $y_2 = f(x)\cdot y_1$. First, we have 
    \begin{align*}
        y_2' &= f'(x)y_1+f(x)y_1' = f'(x)e^{-2x}-2f(x)e^{-2x}\\
        y_2'' &= f''(x)y_1+2f'(x)y_1'+f(x)y_1'' = f''(x)e^{-2x}-4f'(x)e^{-2x}+4f(x)e^{-2x}.
    \end{align*}
    Plugging this into the original differential equation, we get
    \begin{align*}
        f''(x)e^{-2x}-4f'(x)e^{-2x}+4f(x)e^{-2x} + 4\left(f'(x)e^{-2x}-2f(x)e^{-2x} \right)
        + 4f(x)e^{-2x} &= 0.
    \end{align*}
    Simplifying yields $f''(x) = 0$. Now, let $g(x) = f'(x)$. This means
    $g'(x) = f''(x)$. So, $g'(x) = 0$. This is equivalent to $\frac{dg}{dx} = 0$.
    Using separation of variables, we get $g(x) = C$. So, $f'(x) = C$. This is
    the same as $\frac{df}{dx} = C \implies f(x) = C_1x+C_2$. Thus,
    $$\boxed{y_2 = \left(C_1x+C_2\right)e^{-2x}}.$$
    Adding the two solutions, we obtain
    \begin{align*}
        y_h &= y_1 + y_2 = C_0e^{-2x} + C_1 xe^{-2x} +C_2e^{-2x} \\
        y_h &= Ce^{-2x} + Dxe^{-2x}.
    \end{align*}
\end{example}
\begin{theorem}[General Reduction of Order]
    Consider a general homogeneous linear ODE:
    $$y''(x)+p(x)y'(x)+q(x)y(x)=0$$
    and suppose $y_1(x)$ is one solution to the differential equation. Then
    the second solution to the differential equation is
    $$y_2(x) = y_1(x)\int \frac{u(x)}{[y_1(x)]^2} \dif x,$$
    where $u(x) = e^{-\int p(x) \dif x}$.
\end{theorem}
\begin{proof}
    We assume the second solution is of the form $y_2(x) = f(x)y_1(x)$ for
    some function $f$. Thus,
    $$y_2' = f'y_1+fy_1'~~~~y_2'' = f''y_1+2f'y_1'+fy_1''.$$
    Substituting these expressions back into the original differential
    equation, we get
    $$f''y_1+2f'y_1'+fy_1''+p(f'y_1+fy_1')+qfy_1 = 0.$$ Rearranging these
    terms, we obtain $$f(y_1''+py_1'+qy_1)+f''y_1+2f'y_1'+pf'y_1 = 0.$$ Notice
    that the term in the parentheses becomes $0$ since $y_1$ is a solution
    to the differential equation. Therefore, we are simply left with
    $$f''y_1+2f'y_1'+pf'y_1 = 0.$$ Rearranging and letting $g(x) = f'(x)$,
    $$g'(x) = -g(x)\left(2\frac{y_1'(x)}{y_1(x)}+p(x)\right).$$ Now, we simply
    have a first-order ODE in terms of $g(x)$. It can then be shown that
    $f(x) = \int \frac{u(x)}{[y_1(x)]^2} \dif x$ where 
    $u(x) = e^{-\int p(x) \dif x}$. It then follows that
    $$y_2(x) = y_1(x)\int \frac{u(x)}{[y_1(x)]^2} \dif x.$$
\end{proof}
\begin{corollary}[Reduction of Order with Constant Coefficients]
    Consider a homogeneous linear ODE with constant coeffiecients:
    $$ay''(x)+by'(x)+cy(x) = 0$$ where $a,b,c \in \RR \text{\textbackslash} \{0\}$ and suppose the 
    discriminant, $b^2-4ac$, vanishes due to the presence of a repeated
    root in the characteristic equation. Additionally suppose $y_1(x)$ is 
    one solution to the differential equation. Then the second solution to the
    differential equation is $y_2(x) = xy_1(x)$.
\end{corollary}
\begin{proof}
    Letting $y=e^{mx}$ and solving for $m$ from the resulting characteristic
    equation gives $m = -\frac{b}{2a}$. Therefore $y_1(x) = e^{-\frac{b}{2a}x}$.
\end{proof}
\begin{example}
    $$y''+y'+(1/4)y=0;~y(0) = 3; ~ y'(0) = -3.5.$$
\end{example}
\begin{soln}
    Let $y=e^{mx}$. This means $y' = me^{mx}$ and $y''=m^2e^{mx}$. Plugging in,
    we get
    \begin{align*}
        4m^2e^{mx} + 4me^{mx} + e^{mx} &= 0 \\
        e^{mx}(4m^2+4m+1) &=0 \\
        e^{mx}(2m+1)^2 &=0.
    \end{align*}
    Thus, $\boxed{y_1 = C_0e^{-\frac{1}{2}x}}$. We now proceed to use reduction
    of order. So, $y_2 = f(x) \cdot y_1$. First, we have
    \begin{align*}
    y_2' &= f'(x)y_1+f(x)y_1' = f'(x)e^{-x/2}-\frac{1}{2}f(x)e^{-x/2}\\
    y_2'' &= f''(x)y_1+2f'(x)y_1'+f(x)y_1'' = f''(x)e^{-x/2}-f'(x)e^{-x/2}+\frac{1}{4}f(x)e^{-x/2}.
    \end{align*}
    Plugging this into the original differential equation, we get
    \begin{align*}
        f''(x)e^{-x/2}-f'(x)e^{-x/2}+\frac{1}{4}f(x)e^{-x/2} + 
        f'(x)e^{-x/2}-\frac{1}{2}f(x)e^{-x/2} +\frac{1}{4}f(x)e^{-x/2}= 0 \\
        f''(x) = 0.
    \end{align*}
    Now, let $g(x) = f'(x)$. This means that $g'(x) = f''(x)$. So, $g'(x) =0$.
    This means $g(x) = C$. So, $f'(x) =C$. Thus, $f(x) = C_1x+C_2$. THerefore,
    $$y_2 = (C_1x+C_2)e^{-x/2}.$$
    Adding the two solutions, we obtain 
    \begin{align*}
        y_h &= y_1 + y_2 = C_0e^{-x/2} + C_1xe^{-x/2} + C_2e^{-x/2} \\
            &= Ce^{-x/2}+Dxe^{-x/2}.
    \end{align*}
    Using the inital values, we obtain
    $$y_h = 3e^{-x/2}-2xe^{-x/2}.$$
\end{soln}
In general $y_2 = xy_1$.

\begin{example}
    $$y'''+3y''-4y = 0.$$
\end{example}
\begin{soln}
    Let $y=e^{mx}$. This mean $y' = me^{mx}$ and $y'' = m^2e^{mx}$. Plugging
    in, we get 
    \begin{align*}
        m^3e^{mx} + 3m^2e^{mx} -4e^{mx} &= 0 \\
        e^{mx}\left(m^3+3m^2-4\right) &= 0\\
        e^{mx}(m-1)(m^2+4m+4) &= 0\\
        e^{mx}(m-1)(m+2)^2 &= 0.
    \end{align*}
    Therefore, $y_1 = C_1e^x$, $y_2 = C_2e^{-2x}$, and $y_3 = C_3xe^{-2x}$.
    Thus, $$\boxed{y_h = c_1e^{x} + C_2e^{-2x} + C_3xe^{-2x}}.$$
\end{soln}

\begin{example}[Complex Roots]
   $$ y''+4y = 0.$$
\end{example}
\begin{soln}
    Let $y=e^{mx}$. This means that $y'' = m^2e^{mx}$. Plugging in gives
    $m= \pm 2i$. Thus, $y_h = C_1e^{2ix} + C_2e^{-2ix}$. Recall 
    \alert{Euler's Theorem} $$e^{i\theta} = \cos \theta + i\sin \theta.$$
    Plugging this in gives
    \begin{align*}
        y_h &= C_1(\cos 2x + i\sin 2x) + C_2(\cos 2x  -i\sin  2x) \\
        y_h &= C_1\cos 2x + C_2 \sin 2x.
    \end{align*}
\end{soln}
\begin{example}[Complex Roots]
$$y''+8y'+25y = 0.$$
\end{example}
\begin{soln}
    Let $y=e^{mx}$. This meas that $y' = me^{mx}$ and $y'' = m^2e^{mx}$.
    Plugging this in gives $-4\pm 3i$. Thus, 
    \begin{align*}
        y_h &= C_1e^{(-4+3i)x} + C_2e^{(-4-3i)x} \\
        y_h &= e^{-4x}\left(C_1e^{3ix} + C_2e^{-3ix} \right) \\
        y_h &= e^{-4x}\left(C_1 \cos(3x) + C_2 \sin (3x) \right).
    \end{align*}
\end{soln}

\section{Nonhomogeneous Ordinary Differential Equations}
A \alert{nonhomogeneous} differential equation is of the form
$$a_ny^{(n)} +a_{n-1}y^{(n-1)}+\cdots+a_1y'+a_0y = f(x)$$
where $a_i \in \RR[x]$ for $i \in \{1,2,\dots,i\}$ and $f(x) \neq 0$. 
The general solution to a nonhomogeneous
differential equation is $y = y_h+y_p$, where $y_h$ is found from letting
the left hand side equal $0$ and the particular solution, $y_p$, is found
by using one of two methods, namely \alert{undetermined coefficients} or
\alert{variation of parameters}. Undetermined coefficients can only be used
with differential equations with scalar coefficients while variation of parameters
can be used with scalar or variable coefficients.

\begin{theorem}[Undetermined Coefficients]
    Consider the differential equation
    $$a_2y''(x)+a_1y'(x)+a_0y(x) = f(x).$$
    In order to solve for the particular solution, $y_p$, we guess $y_p$
    based on the follwing table
    \begin{center}
        \begin{tabular}{|c|c|}
        \hline
        $f(x)$ & $y_p(x)$ \\
        \hline
        $ae^{\beta x}$ & $Ae^{\beta x}$ \\
        $a\cos(\beta x)$ & $A\cos(\beta x)+B\sin(\beta x)$ \\
        $b\sin(\beta x)$ & $A\cos(\beta x)+B\sin(\beta x)$ \\
        $a\cos(\beta x)+b\sin(\beta x)$ & $A\cos(\beta x)+B\sin(\beta x)$ \\
        $n$th deg polynomial & $A_nx^n+A_{n-1}x^{n-1}+\cdots+A_0$ \\
        \hline
        \end{tabular}
    \end{center}
    In addition, if $f(x)$ is a linear combination of any of the forms
    above, then $y_p(x)$ is also a linear combination of the guesses for
    each of the forms present in $f(x)$.
\end{theorem}

\begin{example}[Undetermined Coefficients]
    $$y''-4y' + 3y = e^{-x};~y(0) = 1;~y'(0)=0.$$
\end{example}
\begin{soln}
    We first find $y_h$ using $y''-4y'+3y = 0$. Thus we obtain
    $$y_h = C_1e^x+C_2e^{3x}.$$ Our guess for $y_p = Ce^{-x}$.
    Thus, $y_p' = -Ce^{-x}$ and $y_p'' = Ce^{-x}$. So, plugging this in,
    we get 
    \begin{align*}
        Ce^{-x}+4Ce^{-x}+3Ce^{-x} = e^{-x} &\implies 8Ce^{-x} = e^{-x} \\
                                           &\implies C = \frac{1}{8}.
    \end{align*}
    Therefore $$y = y_h + y_p = C_1e^{x}+C_2e^{3x} + \frac{1}{8}e^{-x}.$$
    Using the initial values, we find that
    $$\boxed{y = \frac{5}{4}e^{x}-\frac{3}{8}e^{3x} + \frac{1}{8}e^{-x}}.$$
\end{soln}
\begin{example}[Guessing Practice with Undetermined Coefficients]
   $$ y''+2y'+y = \cos(2x)$$
\end{example}
\begin{soln}
    We first proceed to find $y_h$. We find that
    $$y_h = C_1e^{-x}+C_2xe^{-x}.$$
    We make the following guess: $y_p = a\cos(2x) + b\sin(2x)$. Solving
    for $a$ and $b$, we find that $y_p = -\frac{3}{25}\cos(2x) + \frac{4}{25}\sin(2x)$.
    Therefore 
    $$\boxed{y = C_1e^{-x}+C_2xe^{-x} - \frac{3}{25}\cos(2x)+\frac{4}{25}\sin(2x)}.$$
\end{soln}
\begin{example}[Bad Guessing]
    $$y''+4y'+3y = e^{-x}.$$
\end{example}
\begin{soln}
    We notice $y_h = C_1e^{-x} + C_2e^{-3x}$. $y_p = Ce^{-x}$ is a bad guess since it is a term present in the
    homogeneous equation. So, we let $y_p = Cxe^{-x}$, which is a guess from
    the result of applying the reduction of order. Solving for $C$, we get
    that $C = \frac{1}{2}$. Therefore 
    $$\boxed{y = C_1e^{-x}+C_2e^{-3x}+\frac{1}{2}xe^{-x}}.$$
\end{soln}
\begin{theorem}[Variation of Parameters]
    The method of variation of parameters is a technique for finding a particular 
    solution to a nonhomogeneous linear second order ODE:
    $$y''+P(x)y'+Q(x)y = R(x)$$
    provided that the general solution of the corresponding homogeneous linear 
    second order ODE:
    $$y''+P(x)y'+Q(x)y = 0$$ is already known. The particular solution is then
    $$y_p = u(x)y_1(x)+v(x)y_2(x)$$ such that
    $$u'(x) = -\frac{y_2R(x)}{W(y_1,y_2)}~\text{and}~v'(x) = \frac{y_1R(x)}{W(y_1,y_2)},$$
    where $W(y_1,y_2)$ denotes the Wronskian of $y_1$ and $y_2$.
\end{theorem}
\begin{proof}
    First, we have
    \begin{align*}
        y_p' &= \left(u'y_1+uy_1'\right) + \left(v'y_2+vy_2'\right) \\
             &= \left(uy_1'+vy_2'\right) + \left(u'y_1+v'y_2\right)
    \end{align*}
    by the product rule.
\end{proof}
\begin{example}[Variation of parameters]
    $$y''+4y'+4y = e^{-2x}/x^2.$$
\end{example}
\begin{soln}
    We find that $y_h = C_1e^{-2x} + C_2xe^{-2x}$. Notice that the Wronskian
    is $$w =\det \left( \begin{bmatrix} e^{-2x} & xe^{-2x} \\ -2e^{-2x} & e^{-2x}-2xe^{-2x}
    \end{bmatrix}\right) = e^{-4x}.$$ Thus,
    \begin{align*}
        u'(x) &= -\frac{1}{x} \\
        v'(x) &= \frac{1}{x^2}.
    \end{align*}
    Thus, $u(x)=-\ln|x|$ and $v(x) = -\frac{1}{x}$. Therefore,
    $$y_p = \ln |x|e^{2x} - e^{-2x} \implies \boxed{y=C_1e^{-2x}+C_2xe^{-2x}+\ln|x|e^{2x}}.$$
\end{soln}
\begin{example}[Undetermined Coefficients]
    $$y''+y'+y = 3x^2+4.$$
\end{example}
\begin{soln}
    We first find that $$y_h = e^{(-1/2)x}\left( C_1\cos(\sqrt{3}x/2)+C_2\sin(\sqrt{3}x/2)\right).$$
    Then we let $y_p = D_1x^2+D_2x+D_3$. Therefore
    \begin{align*}
        2D_1+2xD_1+D_2+D_1x^2+D_2x+D_3 = 3x^2+4 \\
        x^2\left(D_1\right)+x\left(2D_1+D_2\right)+\left(2D_1+D_2+D_3\right)=3x^2+4.
    \end{align*}
    Solving the resulting system yields $D_1 = 3, D_2 = -6, D_3 = 4$.
    Therefore $$y_p = 3x^2-6x+4.$$ So,
    $$\boxed{y = e^{(-1/2)x}\left( C_1\cos(\sqrt{3}x/2)+C_2\sin(\sqrt{3}x/2)\right)+3x^2-6x+4}.$$
\end{soln}

\begin{example}[Variation of parameters]
    $$y''+4y = \tan x.$$
\end{example}
\begin{soln}
    We first find $y_h = C_1\cos(2x)+C_2\sin(2x)$. Now, we notice that
    wronskian is 
    $$w = \det \left( \begin{bmatrix}\cos(2x)&\sin(2x) \\ -2\sin(2x) & 2\cos(2x) \end{bmatrix}\right)
    =2.$$ Thus, $$u'(x) = -\frac{\tan x \sin (2x)}{2} = -\sin^2(x)$$
    and $$v'(x) = \frac{1}{2}\ln |\cos x| - \frac{\cos(2x)}{4} = \frac{\sin(2x)}{2}-\frac{1}{2}\tan x.$$
    Solving, we find $u(x) = \frac{\sin(2x)}{4} - \frac{1}{2}x$. 
    $v(x) = \frac{1}{2}\ln |\cos x| - \frac{\cos 2x}{4}$. Thus,
    $$y_p = -\frac{1}{2}\cos(2x)+\frac{1}{2}\sin(2x)\ln |\cos x|.$$
    So, $$y = C_1\cos(2x)+C_2\sin(2x)-\frac{1}{2}\cos(2x)+\frac{1}{2}\sin(2x)\ln |\cos x|$$
\end{soln}
\begin{exercise}
    Find $y$ for the differential equation $y''+3y'+2y=\frac{1}{1+e^x}$.
\end{exercise}

\section{Ordinary Differential Equations with Variable Coefficients}
\alert{Euler's Equation} is the following
$$a_nx^ny^{(n)} + a_{n-1}x^{n-1}y^{(n-1)} + \cdots a_1xy'+a_0y = f(x).$$
The methods for solving these differential equations are the 
\alert{transformation method}, which changes to an equation with constant
coefficients, and the \alert{direct method}.

\begin{theorem}[Transformation Method]
    Let $x = e^{z}$. Plugging in $e^z$ into $y(x)$, we get $\hat{y}(z)$.
    Thus, $$y' = \frac{\dif y}{\dif x} = \frac{\dif \hat{y}}{\dif z}\frac{\dif z}{\dif x} = 
    \hat{y}' \frac{\dif z}{\dif x} = \hat{y}' \frac{1}{x}.$$ Then, $y''$ follows
    to be $$y'' = \frac{\hat{y}''-\hat{y}'}{x^2}.$$
\end{theorem}
\begin{example}[Transformation Method]
    We seek to solve the differential equation $$x^2y''-xy'+y=0.$$
    Using the transformation method, we get
    $$x^2\left(\frac{\hat{y}''-\hat{y}'}{x^2}\right) 
    - x\left( \frac{\hat{y}'}{x}\right)+\hat{y} = 0.$$ Simplifying,
    we get $$\hat{y}''-2\hat{y}'+\hat{y} = 0.$$ This is now a differential
    equation with constant coefficients, which we know how to solve easily.
    Thus, we let $\hat{y} = e^{mz}$. Then, $\hat{y_h} = Ae^{z}+Bze^{z}$. Putting
    everything in $x$, we get $$y_h = Ax+Bx\ln x.$$
\end{example}
\begin{theorem}[Direct Method]
    Assume $y=x^m$, where $m \in \CC$.
\end{theorem}
\begin{example}[Direct Method]
    We seek to solve the differential equation $$x^2y''-xy'+y = 0.$$
    Letting $y=x^m$, we have $y' = mx^{m-1}$ and $y'' = m(m-1)x^{m-2}$.
    Plugging in, we get $x^m\left(m^2-2m+1\right) = 0$. Therefore, $m=1$,
    meaning that $y_1 = C_1x$
    We now proceed to use reduction of order. Thus, $y_2 = f(x)y_1$.
    So, 
    \begin{align*}
        y_2' &= f'y_1+fy_1'=f'x+f \\
        y_2'' &= f''y_1+2f'y_1'+y_1''=f''x+2f'.
    \end{align*}
    Plugging this back into the original differential equation, we get
    $$x^2\left(xf''(x)+2f'(x)\right)-x\left(xf'(x)+f(x)\right)+xf(x) = 0.$$
    Simplifying and rearranging, we get $$x^3f''(x)+x^2f'(x) = 0.$$ Letting
    $g(x) = f'(x)$, we obtain the following separable differential equation
    $$\frac{1}{g} \frac{\dif g}{\dif x} = -\frac{1}{x} \implies g(x) = \frac{1}{x}.$$
    Thus, $f(x) = \ln x$. Finally, $y_2 = x \ln x$ and $$y_h = Ax+Bx\ln x.$$
\end{example}
%\section{Problems}
%\begin{problem}
%    Find $y$ for the following differential equation and initial values
%    $$y''+4y'+5y = 0; ~ y(0) = 1; ~ y'(0) = -4.$$
%\end{problem}
%\begin{problem}
%    Find $y$ for the following differential equation and initial values
%    $$y''- 4y'+4y = 0; ~ y(0) = 1; ~ y'(0) = 0.$$
%\end{problem}
%\begin{problem}
%    Find $y$ for the following differential equation and initial values
%    $$y'''-7y'+6y = 0; ~ y(0) = 0; ~ y'(0) = 0; ~ y''(0) = 1.$$
%\end{problem}
 

